\section{Experimental Evaluation}
\label{eval}

\subsection{Datasets and evaluation protocol} \label{eval:protocol}
\subsubsection{Holidays and Oxford5K.} We use two standard image retrieval datasets for our experiments. The first one is the INRIA \emph{Holidays} dataset \cite{holidays}. This dataset consists of 1491 images divided into 500 groups of matching images. %If the dataset of an experiment is not mentioned in what follows this means, by default, that it is performed on \emph{Holidays}.
The second one is  the \emph{Oxford5k} dataset \cite{oxford}, which consists of 5062 images divided into 55 groups of matching images. 

In both datasets, each group contains one query image, the other images in the group being the only correct answers to the query. We use the mAP computation procedures provided along with the dataset to evaluate performance.

\subsubsection{Flickr $1e6$.} Besides these two datasets, we also use $1e6$ random distractor images downloaded from \emph{Flickr} to carry out large scale experiments, following standard protocol \cite{holidays,Delhumeau2013,ZePe15}.

%For each query image, we calculate its similarity to all other images in the database and rank them, in decreasing order.
%The average precision of a group is derived from the ranking of the images of the group for the similarity with the corresponding query image.
%The final mean average precision (mAP) for a dataset is the mean of the average precision over all its groups.
\subsubsection{Negative pool.} As a pool of negative images to build E-SVM and SLEM representations, we use the Flickr100k collection \cite{oxford}, composed of $10^5$ random Flickr images disjoint from those in Flickr $1e6$.. For a full rank decomposition, we use only a subset of it, containing between 6000 and 15000 images.
As stated in section \ref{low-rank} and further discussed in section \ref{time-scale}, a full rank decomposition does not scale well for bigger number of negative samples. For low rank decomposition, we use all 100000 images.


At evaluation time, for a dataset that consists of $p$ images and $q$ query images, we calculate its $p\times q$ \emph{similarity matrix} $S$, where each of its $q$ columns is the matching scores of the query image with all the $p$ images.


\subsection{Which kernel to choose?}
For the experiment of SLEM as a feature encoder, we tested three different kernels --linear, Gaussian, polynomial and Spatial Pyramid Matching (SPM) -- defined below:
\begin{align}
    k_1(x,y) = x^Ty; \
    k_2(x,y) = e^{-\gamma\|x-y \|^2}; \ 
    k_3(x,y) = x^Ty+\gamma(x^Ty)^2. \label{kernels}
\end{align}
%\JZ{We give results with spp, we should add it. Makes it clear that we use a wide range of kernels.}
%\JZ{The text below could be removed. But some brief text for SPM is a good idea, as it is less standard. I think SPP and SPM are not the same thing... We should use SPM... SPP is a finite dimensional feature map.}
% \textbf{Linear SLEM} The linear kernel of Equation (\ref{k:lin}) is the first default choice and equivalent to the non-kernelized version of SLEM.
% In the remaining of this paper, we reference to linear SLEM when we use the non-kernelized SLEM.
% \textbf{Gaussian SLEM} The radial basis function kernel of Equation (\ref{k:rbf}) is a
% well known reproducing kernel, used for classification with support vector machines.
% \textbf{Polynomial SLEM} The polynomial kernel of Equation (\ref{k:poly}) is a reproducing kernel often used in natural language processing.
%\textbf{SPM SLEM} The spatial pyramid matching kernel of $L+1$ levels in Equation \ref{k:spm} take as input a set of local descriptors and its location in pyramidal bins \cite{spk}. 
Note that Gaussian and polynomial kernels require a scalar parameter ($\gamma$ and $\lambda$, respectively), that we set using cross-validation \JZ{What is the cross-valid dataset?}.
The SPM kernel of $L+1$ levels in Equation \eqref{k:spm} takes as input a set of local descriptors and computes a similarity score that takes into account their geometrical layout \cite{spk}.\JZ{Need to define $H_x^{l,\gamma}$ ...}
%its location in pyramidal bins \cite{spk}. 


\subsection{Base visual features}
We test our feature encoder for three different base features $x\in\mathbb{R}^d$ the hand-crafted VLAD image representation \cite{Delhumeau2013} and two learned features derived from the activation coefficients of deep Convolutional Neural Networks \cite{Krizhevsky2012,babenko15}.

We use the same VLAD variant of \cite{Delhumeau2013} used in \cite{ZePe15} that relies on densely-extracted rootSIFT \cite{3things} local descriptors, per-cluster, PCA-based rotations, and root normalization. Like \cite{ZePe15}, we use $64$ clusters, for a final feature size of $8192$.

The first CNN-based feature we use consists of the activation coefficients of the previous-to-last layer of the architecture of \cite{Krizhevsky2012}, based on a publicly available pre-trained model \cite{jia2014caffe}. This is the same feature considered in \cite{ZePe15} and originally considered as an image retrieval feature in \cite{Sharif}. We refer to it as FC-CNN. \JZ{Please verify...}

The second CNN-based feature, the SPoC feature of \cite{babenko15}, is tailored specifically for the image retrieval application. It consists of a spatially-weighted sum-pooling of the activations of the last convolutional layer of a 19-layer CNN \cite{Simonyan2014}.

%REMOVED
% We revisit the VLAD feature presented in \cite{VLAD} as an example of a hand-crafted representation. First we extract a set $\mathcal{F}$ of local descriptors of the image $I$. We use the 128 dimension RootSIFT \cite{3things} descriptors, extracted densely.
% Then, we hard-assign each descriptor $f$ in $\mathcal{F}$ to the closest among $K$ pre-trained codewords $c_k$, $k\in\{1\cdots K\}$,
% %one of $K$ set $\mathcal{C}_k$ of descriptors associated to codewords $\{c_k\}_{1\leq k\leq K}$, 
% and map $f$ to a $\RR^{128K}$ vector
% \begin{equation}
% \phi^{VL}_1(f) = \left[0 \cdots 0\quad \Phi_k^T\frac{(f-c_k)}{\|f-c_k\|} \quad 0 \cdots 0\right],
% \end{equation}
% where $\Phi_k$ is a $128\times 128$ PCA matrix learned on training features mapped to $k$-the codeword.
% %associated to descriptors in $\mathcal{C}_k$. 
% The final VLAD representation is the power-normalization and $l_2$ normalization of the sum-pooling of $\phi^{VL}_1$:
% \begin{equation}
% \phi^{VL}_2(I) \propto \mathrm{power}\big(\sum_{f\in \mathcal{F}}\phi^{VL}_1(f)\big),~\|\phi^{VL}_2(I)\|=1,
% \end{equation}
% with scalar power normalization $\mathrm{power}(v)=\mathrm{sign}(v)|v|^{0.5}$ applied component-wise.
% In experiments, we use $K=64$ codewords learned on images from Flickr. Our VLAD representation has $d=8192$ dimensions.


%REMOVED
% Convolutional features obtained from very deep convolutional neural networks (CNNs) have been shown to work as good local descriptors for matching \cite{SimonZisser15}. The SPoC representation \cite{babenko15} is a weighted sum-pooling of the activations of the last convolutional layer of a 19-layer CNN. For a given input image $I$, the activations in this layer are organized over a $W\times H$ spatial grid and over $d$ channels. Each position $(w,h)\in \{1 \cdots W\}\times \{1\cdots H\}$ can thus be equipped with a descriptor $f_{h,w}(I)\in\RR^D$.
% %Indeed, if our last convolutional layer has $D$ neurons, and each neuron a $W\times H$ map of activations of this neurons to the image $I$, 
% %each pair $(w,h)$ with $w$ in $\{1,2,..., W\}$ and $h$ in $\{1,2,...,H\}$ can be associated to a descriptor $f_{(h,w)}$ in $\RR^D$ of the responses of each neuron at coordinate $(w,h)$ of the maps. 
% We then sum-pool these descriptors, weighted accordingly to their distance to the center of the image:
% \begin{equation}
%     \phi^{SPoC}(I) = \sum_{w=1}^W\sum_{h=1}^H \alpha_{w,h}f_{w,h}(I),
% \end{equation}
% where
% \begin{equation}
%     \alpha_{w,h} = \exp \left(-\dfrac{(w-W/2)^2+(h-H/2)^2}{2\sigma^2}\right).
% \end{equation}
% In our experiments, we follow the implementation details of \cite{babenko15}: We resize all images to $586\times 586$ pixels before feeding them to the network. The last convolutional layer has $d=512$ channels with activation maps of size $(W,H)=(37,37)$. We also set $\sigma=\frac{H}{3}$.

% Finally, we also test a more convencional CNN feature, less deep and using two fully connected layers after the convolutional layers.
% Our final representation is a $4096$ non-negative feature. We based out implementation on CAFFE \cite{jia2014caffe}.


% \subsection{Implementation details}
% We cross validate $\gamma$ and $\lambda$ for Poly and Gaussian SLEM.

\subsection{Full rank results}
In Table \ref{fullrank:results}, we test our method using tye three recent base features discussed above (VLAD \cite{Delhumeau2013}, SPoC \cite{babenko15}, and FC-CNN \cite{Sharif}). We further compare against the ESVM method of \cite{ZePe15}, as well as the PCA+whitening strategy shown in \cite{babenko15} to drastically improve the retrieval results for SPoC. In our experiments, using PCA dimensionality reduction worsens the results, and hence we use the full PCA projection matrix for the PCA+whitening strategy.
%Hence we compare the improvements of PCA plus whitening, without compression, with our method.

Linear SLEM performs similarly to ESVM despite being much more time efficient (see discussion in section \ref{time-scale}).
Gaussian SLEM and Polynomial SLEM outperform all methods, for all datasets and all image representations.
The results are presented in Table \ref{fullrank:results}.

\begin{table*}[t]
\begin{center}
\caption{Mean average precision results for INRIA Holidays and Oxford buildings datasets, expressed as percentages. In this table, we present our results for VLAD-64 \ref{VLAD}, sum-pooling of convolutional features (SPoC) \ref{babenko15} and fully connected (fc) CNN \ref{jia2014caffe}}
\begin{tabular}{|c|c|c|c|c|c|c|c|}
\hline
Dataset & \multicolumn{3}{|c|}{\textbf{Holidays}} & \multicolumn{3}{|c|}{\textbf{Oxford 5k}}\\
\hline
Method, features & VLAD-64  & SPoC CNN & FC CNN & VLAD-64 & SPoC CNN & FC CNN\\
\hline\hline
Baseline            & 72.7         & 73.1         & 68.2         & 46.3           & 54.4         & 40.6\\
%Whitening           & -    & -    & -    & -    & -\\
PCA+whitening       & 69.4         & 77.5         & 69.2         & 50.9           & 63.7         & 45 \\
E-SVM               & \textbf{78.3} & 79.9         & 71.8         & 57.5           & 62.1         & 44.6\\
Linear SLEM         & 78           & 78.3         & 72.1         & \textbf{59.3}   & 64.1         & 45.5\\
Gaussian SLEM       & 78.1         & 81.4         & \textbf{72.9} & 59             & \textbf{64.9} & 46.1\\
Poly SLEM           & 78.1         & \textbf{82}   & \textbf{72.9} & \textbf{59.3}  & 64.8         & 46\\
\hline
\end{tabular}
\end{center}
\label{fullrank:results}
\end{table*}

\begin{figure}[!h]
\centering
\begin{tikzpicture}
	\begin{semilogxaxis}[
		xlabel=Number of distractors,
		ylabel=mAP,
		legend = south west]
%%Poly SLEM
    \addplot[PolySLEM] coordinates{
        (   0, 0.5929)
        (1000, 0.5918)
        (3000, 0.5901)
        (6000, 0.5881)
       (10000, 0.5864)
       (30000, 0.5759)
       (60000, 0.5663)
      (100000, 0.5598)
      (300000, 0.5405)
      (600000, 0.5310)
    };
    \addlegendentry{PolySLEM}
	\addplot[ESVM] coordinates{
        (   0, 0.5750)
        (1000, 0.5738)
        (3000, 0.5728)
        (6000, 0.5715)
       (10000, 0.5700)
       (30000, 0.5620)
       (60000, 0.5533)
      (100000, 0.5477)
      (300000, 0.5276)
      (600000, 0.5121)
    };	
    \addlegendentry{RESVM-2}
    %\addplot[VLAD] coordinates{
    %    (   0, 0.463)
    %    (1000, 0.4611)
    %    (3000, 0.4585)
    %    (6000, 0.4558)
    %   (10000, 0.4534)
    %   (30000, 0.4443)
    %   (60000, 0.4372)
    %  (100000, 0.4322)
    %  (300000, 0.4170)
    %  (600000, 0.4050)
    %};
    %\addlegendentry{VLAD}
    \addplot[LinSLEM] coordinates{
        (   0, 0.5933)
        (1000, 0.5922)
        (3000, 0.5904)
        (6000, 0.5883)
       (10000, 0.5860)
       (30000, 0.5751)
       (60000, 0.5643)
      (100000, 0.5588)
      (300000, 0.5385)
      (600000, 0.5260)
    };
    \addlegendentry{LinSLEM}
	\end{semilogxaxis}
\end{tikzpicture}
\caption{mAP for Oxford 5k varying the number of distractors, using VLAD features and different methods of SLEM (see legend).}
\label{vlad:oxford}
\end{figure}


\subsection{Time Scalability} \label{time-scale}
In this section we compare the time efficiency of our method and the E-SVM, as well as discuss which method to use accordingly with the number of positive and negative samples.

In Figure \ref{fullrank:results}, we see Linear SLEM efficiency does not change with $n$.
Indeed, if $d$ is the dimension of the base representation, $A$ is a $d\times d$ matrix for Linear SLEM, whereas for a full rank kernel, $A$ is a $n\times n$.
This explain the increasing running time for Gaussian and polynomial kernels: storage and solving a $n\times n$ system does not scale for large number of negative samples.
But retrieval results in Figure \ref{fullrank:results} suggest we can benefit from larger sets of negative samples. Linear SLEM and ESVM scales well, but do not perform so well.

\vspace{3 mm}

\begin{figure}
  \begin{tikzpicture}
    \begin{groupplot}
      [group style={%
        columns=2,
        rows=2,
        group name=plots,
        xlabels at=edge bottom,
        %y descriptions at=all,
        horizontal sep=5em,        
      },
      % ybar,
      % ymin=0,
      % ymax=27e3,
      enlarge x limits={abs=.5},
      width=0.5\textwidth,
      height=0.4\textwidth,
      % scaled y ticks=base 10:-3,
      % xticklabels from table={\first}{Criterion},
      % x tick label style={rotate=90,anchor=east},
      % xtick=data,
      ]

      \nextgroupplot[ylabel=mAP]%xlabel=Num. of negatives $n$, ylabel=mAP]
      ]
      %% Poly SLEM
      \addplot[PolySLEM] coordinates {
        (500,  0.81874)
        (1500, 0.83777)
        (2500, 0.83877)
        (3500, 0.84462)
        (4500, 0.84942)
        (5500, 0.8544)
        (6500, 0.85363)
        (7500, 0.85724)
        (8500, 0.85574)
        (9500, 0.86146)
        (10500,0.86158)
        (11500,0.86131)
        (12500,0.8619)
        (13500,0.86165)
        (14500,0.86309)
      };
      %% Gaussian SLEM
      \addplot[GaussSLEM] coordinates {
        (500,  0.82012)
        (1500, 0.83678)
        (2500, 0.83791)
        (3500, 0.84470)
        (4500, 0.84844)
        (5500, 0.85039)
        (6500, 0.85423)
        (7500, 0.85568)
        (8500, 0.85361)
        (9500, 0.85254)
        (10500,0.85393)
        (11500,0.85398)
        (12500,0.85513)
        (13500,0.85744)
        (14500,0.86092)
      };
      %% ESVM8
      \addplot[ESVM] coordinates {
        (500,  0.81446)
        (1500, 0.82524)
        (2500, 0.82707)
        (3500, 0.82359)
        (4500, 0.82719)
        (5500, 0.82689)
        (6500, 0.82993)
        (7500, 0.83019)
        (8500, 0.82972)
        (9500, 0.83280)
        (10500,0.83289)
        (11500,0.83457)
        (12500,0.83421)
        (13500,0.83968)
        (14500,0.83995)
      };
      %% Linear SLEM
      \addplot[LinSLEM] coordinates {
        (500,  0.81524)
        (1500, 0.81843)
        (2500, 0.82042)
        (3500, 0.81973)
        (4500, 0.81862)
        (5500, 0.81760)
        (6500, 0.82291)
        (7500, 0.82330)
        (8500, 0.82092)
        (9500, 0.82221)
        (10500,0.81929)
        (11500,0.82335)
        (12500,0.82301)
        (13500,0.82254)
        (14500,0.82311)
      };

      \nextgroupplot[
      ymode=log,
      %xlabel=Num. of negatives $n$,
      ylabel=Time per image (s),
      legend to name=grouplegend,
      legend style={legend columns=-1},
      % legend style={at={(0.465,-0.45)},
      % anchor=north,legend columns=-1},
      ]%
      %% Poly SLEM
      \addplot[PolySLEM] coordinates {
        (500,  0.0002099)
        (1500, 0.0009621)
        (2500, 0.0026628)
        (3500, 0.0054810)
        (4500, 0.0086532)
        (5500, 0.0121226)
        (6500, 0.0188800)
        (7500, 0.0228739)
        (8500, 0.0434941)
        (9500, 0.0499160)
        (10500,0.0477895)
        (11500,0.0703987)
        (12500,0.0842496)
        (13500,0.0802379)
        (14500,0.1245190)
      };
      \addlegendentry{Poly SLEM}
      %% Gaussian SLEM
      \addplot[GaussSLEM] coordinates {
        (500,  0.0006671)
        (1500, 0.0012690)
        (2500, 0.0027834)
        (3500, 0.0055151)
        (4500, 0.0094601)
        (5500, 0.0140014)
        (6500, 0.0184553)
        (7500, 0.0249103)
        (8500, 0.0430012)
        (9500, 0.0539018)
        (10500,0.0617439)
        (11500,0.0695579)
        (12500,0.0865806)
        (13500,0.1049192)
        (14500,0.1037325)
      };
      \addlegendentry{Gaussian SLEM}
      %% ESVM
      \addplot[ESVM] coordinates {
        % (600, .000610)
        % (6000, .00427)
        % (60000, .0378)
        (1000 * 6/10, .000610)
        (10000* 6/10, .00427)
        (14500, .0095478) %Interpolated
        % (100000 * 6/10, .0378)
        % (500,  0.014)
        % (1500, 0.015)
        % (2500, 0.016)
        % (3500, 0.017)
        % (4500, 0.018)
        % (5500, 0.019)
        % (6500, 0.02)
        % (7500, 0.021)
        % (8500, 0.022)
        % (9500, 0.023)
        % (10500,0.024)
        % (11500,0.025)
        % (12500,0.026)
        % (13500,0.027)
        % (14500,0.028)
      };
      \addlegendentry{ESVM}
      %% Linear SLEM
      \addplot[LinSLEM] coordinates {
        (500,  0.0000622)
        (1500, 0.0000704)
        (2500, 0.0000581)
        (3500, 0.0000592)
        (4500, 0.0000902)
        (5500, 0.0000655)
        (6500, 0.0001069)
        (7500, 0.0000973)
        (8500, 0.0000914)
        (9500, 0.0000812)
        (10500,0.0000923)
        (11500,0.0001047)
        (12500,0.0001122)
        (13500,0.0001251)
        (14500,0.0001156)
      };
      \addlegendentry{Linear SLEM}
      \addplot[mark=*, green, dashed] coordinates {
            (500,  0.0001208)
            (1500, 0.0003114)
            (2500, 0.0008255)
            (3500, 0.0016757)
            (4500, 0.0017118)
            (5500, 0.0025198)
            (6500, 0.0033489)
            (7500, 0.0042703)
            (8500, 0.0060299)
            (9500, 0.0074267)
            (10500,0.0094138)
            (11500,0.0113405)
            (12500,0.0121783)
            (13500,0.0144320)
            (14500,0.0171987)
          };
          \addplot[mark=*, blue, dashed] coordinates {
            (500,  0.0001501)
            (1500, 0.0003122)
            (2500, 0.0006261)
            (3500, 0.0011822)
            (4500, 0.0015350)
            (5500, 0.0023396)
            (6500, 0.0047356)
            (7500, 0.0051036)
            (8500, 0.0061753)
            (9500, 0.0092072)
            (10500,0.0101006)
            (11500,0.0117932)
            (12500,0.0154758)
            (13500,0.0185985)
            (14500,0.0194114)
          };
      
      

    \nextgroupplot[xlabel=Num. of negatives $n$,
      ylabel=mAP]
      ]
      %% Poly SLEM
      \addplot[PolySLEM] coordinates {
        (500,  0.7063)
        (1500, 0.7137)
        (2500, 0.7118)
        (3500, 0.7147)
        (4500, 0.7152)
        (5500, 0.7167)
        (6500, 0.7182)
        (7500, 0.7189)
        (8500, 0.7198)
        (9500, 0.7202)
        (10500,0.7210)
        (11500,0.7219)
        (12500,0.7224)
        (13500,0.7223)
        (14500,0.7215)
      };
      %% Gaussian SLEM
      \addplot[GaussSLEM] coordinates {
        (500,  0.7054)
        (1500, 0.7130)
        (2500, 0.7150)
        (3500, 0.7156)
        (4500, 0.7162)
        (5500, 0.7172)
        (6500, 0.7174)
        (7500, 0.7185)
        (8500, 0.7194)
        (9500, 0.7193)
        (10500,0.7202)
        (11500,0.7197)
        (12500,0.7206)
        (13500,0.7205)
        (14500,0.7206)
      };
      %% ESVM8
      \addplot[ESVM] coordinates {
        (500,  0.71)
        (1500, 0.71)
        (2500, 0.71)
        (3500, 0.71)
        (4500, 0.71)
        (5500, 0.71)
        (6500, 0.71)
        (7500, 0.71)
        (8500, 0.71)
        (9500, 0.71)
        (10500,0.71)
        (11500,0.71)
        (12500,0.71)
        (13500,0.71)
        (14500,0.71)
      };
      %% Linear SLEM
      \addplot[LinSLEM] coordinates {
        (500,  0.7062)
        (1500, 0.7137)
        (2500, 0.7118)
        (3500, 0.7148)
        (4500, 0.7154)
        (5500, 0.7168)
        (6500, 0.7183)
        (7500, 0.7188)
        (8500, 0.7199)
        (9500, 0.7202)
        (10500,0.7211)
        (11500,0.7219)
        (12500,0.7224)
        (13500,0.7224)
        (14500,0.7215)
        
      };

      \nextgroupplot[
      ymode=log,
      xlabel=Num. of negatives $n$,
      ylabel=Time per image (s),
      legend to name=grouplegend,
      legend style={legend columns=-1},
      % legend style={at={(0.465,-0.45)},
      % anchor=north,legend columns=-1},
      ]%
      %% Poly SLEM
      \addplot[PolySLEM] coordinates {
        (500,  0.0002099)
        (1500, 0.0009621)
        (2500, 0.0026628)
        (3500, 0.0054810)
        (4500, 0.0086532)
        (5500, 0.0121226)
        (6500, 0.0188800)
        (7500, 0.0228739)
        (8500, 0.0434941)
        (9500, 0.0499160)
        (10500,0.0477895)
        (11500,0.0703987)
        (12500,0.0842496)
        (13500,0.0802379)
        (14500,0.1245190)
      };
      \addlegendentry{Poly SLEM}
      %% Gaussian SLEM
      \addplot[GaussSLEM] coordinates {
        (500,  0.0006671)
        (1500, 0.0012690)
        (2500, 0.0027834)
        (3500, 0.0055151)
        (4500, 0.0094601)
        (5500, 0.0140014)
        (6500, 0.0184553)
        (7500, 0.0249103)
        (8500, 0.0430012)
        (9500, 0.0539018)
        (10500,0.0617439)
        (11500,0.0695579)
        (12500,0.0865806)
        (13500,0.1049192)
        (14500,0.1037325)
      };
      \addlegendentry{Gaussian SLEM}
      %% ESVM
      \addplot[ESVM] coordinates {
        % (600, .000610)
        % (6000, .00427)
        % (60000, .0378)
        (1000 * 6/10, .000610)
        (10000* 6/10, .00427)
        (14500, .0095478) %Interpolated
        % (100000 * 6/10, .0378)
        % (500,  0.014)
        % (1500, 0.015)
        % (2500, 0.016)
        % (3500, 0.017)
        % (4500, 0.018)
        % (5500, 0.019)
        % (6500, 0.02)
        % (7500, 0.021)
        % (8500, 0.022)
        % (9500, 0.023)
        % (10500,0.024)
        % (11500,0.025)
        % (12500,0.026)
        % (13500,0.027)
        % (14500,0.028)
      };
      \addlegendentry{ESVM}
      %% Linear SLEM
      \addplot[LinSLEM] coordinates {
        (500,  0.0000622)
        (1500, 0.0000704)
        (2500, 0.0000581)
        (3500, 0.0000592)
        (4500, 0.0000902)
        (5500, 0.0000655)
        (6500, 0.0001069)
        (7500, 0.0000973)
        (8500, 0.0000914)
        (9500, 0.0000812)
        (10500,0.0000923)
        (11500,0.0001047)
        (12500,0.0001122)
        (13500,0.0001251)
        (14500,0.0001156)
      };
      \addlegendentry{Linear SLEM}
      \addplot[mark=*, green, dashed] coordinates {
            (500,  0.0001208)
            (1500, 0.0003114)
            (2500, 0.0008255)
            (3500, 0.0016757)
            (4500, 0.0017118)
            (5500, 0.0025198)
            (6500, 0.0033489)
            (7500, 0.0042703)
            (8500, 0.0060299)
            (9500, 0.0074267)
            (10500,0.0094138)
            (11500,0.0113405)
            (12500,0.0121783)
            (13500,0.0144320)
            (14500,0.0171987)
          };
          \addplot[mark=*, blue, dashed] coordinates {
            (500,  0.0001501)
            (1500, 0.0003122)
            (2500, 0.0006261)
            (3500, 0.0011822)
            (4500, 0.0015350)
            (5500, 0.0023396)
            (6500, 0.0047356)
            (7500, 0.0051036)
            (8500, 0.0061753)
            (9500, 0.0092072)
            (10500,0.0101006)
            (11500,0.0117932)
            (12500,0.0154758)
            (13500,0.0185985)
            (14500,0.0194114)
          };
      
      

    \end{groupplot}

    \node at (plots c1r1.north east) [anchor=south, xshift=2.5em] {\ref{grouplegend}};
    %\draw (plots c2r1.north west) circle (3pt) node {North west};

  \end{tikzpicture}
  \caption{Comparison of results for different methods of SLEM (see legend). Above, results for INRIA Holidays, using SPoC features; Below, results for Oxford 5k using NetVLAD features. We use $T=10^5$ iterations for all $n$ to report mAP for ESVM, as suggested by \cite{ZePe15}, but report timings using $T=1.66 n$ and the values reported in Table 1 of \cite{ZePe15}. Left: mAP; Right: computation time in solid line, \emph{online} computational cost on dashed line.}
  \label{fullrank:results}
\end{figure}




% \begin{figure}[!ht]
% \begin{center}
%   \begin{tikzpicture}
%     \begin{axis}[
%           xlabel=Num. of negatives,          
%           ylabel=mAP]
%           %% Poly SLEM
%           \addplot[PolySLEM] coordinates {
%             (500,  0.78852)
%             (1500, 0.80397)
%             (2500, 0.80648)
%             (3500, 0.80905)
%             (4500, 0.81192)
%             (5500, 0.81502)
%             (6500, 0.81659)
%             (7500, 0.81712)
%             (8500, 0.81599)
%             (9500, 0.81432)
%             (10500,0.81625)
%             (11500,0.81796)
%             (12500,0.81726)
%             (13500,0.81993)
%             (14500,0.82057)
%           };
%           %% Gaussian SLEM
%           \addplot[GaussSLEM] coordinates {
%             (500,  0.77757)
%             (1500, 0.79756)
%             (2500, 0.80338)
%             (3500, 0.79909)
%             (4500, 0.79950)
%             (5500, 0.80257)
%             (6500, 0.8016)
%             (7500, 0.80362)
%             (8500, 0.80377)
%             (9500, 0.80501)
%             (10500,0.80534)
%             (11500,0.80909)
%             (12500,0.80955)
%             (13500,0.80813)
%             (14500,0.81214)
%           };
%           %% ESVM
%           \addplot[ESVM] coordinates {
%             (500,  0.7684)
%             (1500, 0.78434)
%             (2500, 0.78836)
%             (3500, 0.78927)
%             (4500, 0.79292)
%             (5500, 0.79523)
%             (6500, 0.79491)
%             (7500, 0.79557)
%             (8500, 0.79915)
%             (9500, 0.79819)
%             (10500,0.79888)
%             (11500,0.7985)
%             (12500,0.79822)
%             (13500,0.79930)
%             (14500,0.79905)
%           };
%           %% Linear SLEM
%           \addplot[LinSLEM] coordinates {
%             (500,  0.77638)
%             (1500, 0.7748)
%             (2500, 0.77919)
%             (3500, 0.77914)
%             (4500, 0.7748)
%             (5500, 0.77578)
%             (6500, 0.77902)
%             (7500, 0.78338)
%             (8500, 0.78043)
%             (9500, 0.78026)
%             (10500,0.77803)
%             (11500,0.77767)
%             (12500,0.77818)
%             (13500,0.78148)
%             (14500,0.78254)
%           };
% 	\end{axis}
%       \end{tikzpicture}
    
%       \begin{tikzpicture}
% 	\begin{semilogyaxis}[
%           xlabel=Num. of negatives,
%           ylabel=Time per image (s),          
%           legend style={at={(0.465,-0.45)},
%             anchor=north,legend columns=-1},
%           ]%legend pos=outer south]
%           %% Poly SLEM
%           \addplot[PolySLEM] coordinates {
%             (500,  0.0001)
%             (1500, 0.0007)
%             (2500, 0.0023)
%             (3500, 0.0046)
%             (4500, 0.0088)
%             (5500, 0.0146)
%             (6500, 0.0213)
%             (7500, 0.0289)
%             (8500, 0.0395)
%             (9500, 0.0542)
%             (10500,0.0703)
%             (11500,0.0903)
%             (12500,0.1103)
%             (13500,0.1434)
%             (14500,0.1745)
%           };
%           \addlegendentry{Poly SLEM}
%           %% Gaussian SLEM
%           \addplot[GaussSLEM] coordinates {
%             (500,  0.0003)
%             (1500, 0.0008)
%             (2500, 0.0021)
%             (3500, 0.0044)
%             (4500, 0.0072)
%             (5500, 0.0119)
%             (6500, 0.0172)
%             (7500, 0.026)
%             (8500, 0.0344)
%             (9500, 0.0465)
%             (10500,0.0617)
%             (11500,0.0814)
%             (12500,0.1248)
%             (13500,0.1301)
%             (14500,0.161)
%           };
%           \addlegendentry{Gaussian SLEM}
%           %% ESVM
%           \addplot[ESVM] coordinates {
%             (1000 * 6/10, .000610)
%             (10000* 6/10, .00427)            
%             (14500, .0095478) %Interpolated
%             % (100000 * 6/10, .0378)
%             % (500,  0.014)
%             % (1500, 0.015)
%             % (2500, 0.016)
%             % (3500, 0.017)
%             % (4500, 0.018)
%             % (5500, 0.019)
%             % (6500, 0.02)
%             % (7500, 0.021)
%             % (8500, 0.022)
%             % (9500, 0.023)
%             % (10500,0.024)
%             % (11500,0.025)
%             % (12500,0.026)
%             % (13500,0.027)
%             % (14500,0.028)
%           };
%           \addlegendentry{ESVM}
%           %% Linear SLEM
%           \addplot[LinSLEM] coordinates {
%             (500,  0.0000565)
%             (1500, 0.0000575)
%             (2500, 0.0000467)
%             (3500, 0.0000679)
%             (4500, 0.0000683)
%             (5500, 0.0000612)
%             (6500, 0.0000608)
%             (7500, 0.0000708)
%             (8500, 0.0000743)
%             (9500, 0.0000996)
%             (10500, 0.0000993)
%             (11500,0.0001073)
%             (12500,0.0001027)
%             (13500,0.000075)
%             (14500,0.0001141)
%           };
%           \addlegendentry{Linear SLEM}
% 	\end{semilogyaxis}
%       \end{tikzpicture}
%       \caption{Results for INRIA Holidays, using SPoC features and different methods of SLEM (see legend).We use $T=10^5$ iterations for all $n$ to report mAP for ESVM, as per \cite{ZePe15}, but report timings using $T=1.66 n$ and the values reported in Table 1 of \cite{ZePe15}. Left: mAP; right: online computation time.}
%       \label{fullrank:results}
%     \end{center}
% \end{figure}

%% Local Variables:
%% TeX-master: "main_eccv"
%% End:

\subsection{Low-rank decomposition evaluation}


%\begin{figure}[!h]
%\centering
%\begin{subfigure}[b]{0.48\textwidth}
%\includegraphics[width=\textwidth]{linear_decomposition_nolog.png}
%\end{subfigure}
%\begin{subfigure}[b]{0.48\textwidth}
%\includegraphics[width=\textwidth]{rbf_decomposition_nolog.png}
%\end{subfigure}
%\caption{Comparison between full rank and low rank. In blue, mAP results for full rank SLEM. %In red, mAP results for low rank decomposition of SLEM, varying the rank $r'$.
%At the left, linear SLEM results;at the right, Gaussian SLEM results. In this experiment, $n=10281$.}
%\label{no.ker.vs.linear2}
%\end{figure}

\subsection{Spatial pyramid matching kernel}

\begin{table}[!h]
    \centering
    \caption{Results for spatial pyramids kernel when compared with a Bag of Visual Words from the local descriptors}
    \begin{tabular}{|c|c|c|c|c|}
    \hline
    Dataset & \multicolumn{2}{|c|}{\textbf{Holidays}} & \multicolumn{2}{|c|}{\textbf{Oxford 5k}}\\
    \hline
        Method, feaures & RootSIFT & CNN & RootSIFT
        &CNN \\
    \hline
    \hline
    Baseline (BoW) & 56.3 & 49.8 & 31.1 & 32.4 \\
        SP SLEM    & 66.9 & 70.2 & 42.4 & 45 \\
    \hline
    \end{tabular}
    \label{tab:spk}
\end{table}
All previous kernel functions take as input image representations in a fixed sized vectorial space. In this section, we propose using the spatial pyramid kernel \cite{GrauDa05}, that takes as input a set of local descriptors and its coordinates in the image.

We revisit the spatial pyramid scheme presented in \cite{spk} using the generalized intersection function of histograms.

Let $X$, $Y$ be the sets of local descriptors of a pair of images.
These local descriptors can be associated to one of $M$ codewords $\{c_1,c_2,...,c_M\}$, so that $X=\bigcup_{1\leq m \leq M}X_m$ and $Y=\bigcup_{1\leq m \leq M}Y_m$, so that $X_m$, $Y_m$ are the set of descriptors associated to $c_m$.
For level $l$ in $\{0,1,..., L\}$, we divide the images in grids $2^l\times 2^l$ and we call $H^{(l,m)}_X$, $H^{(l,m)}_Y$ the $4^l$-dimensional histograms of the occurrences of the of the visual word $c_m$ in each bin of the grid.
The kernel $K_{sp}(X,Y)$ is given by sum over all codewords of the intersection histogram of $H^{(l,m)}_X$ and $H^{(l,m)}_Y$, $l=0,...,L$,  weighted proportionally to the number of bins in each level.

We test this kernel for RootSIFT extracted in multiscales, as used in \cite{spk}, and learn the codewords with K-means. Inspired by the results of \cite{SPPCNN}, we also test it for activations of the last convolutional layer of the network of \cite{SimonZisser15}. Each neuron of the last layer correspont to a codeword. The results are presented in Table \ref{tab:spk}.

%The spatial pyramid kernel is given by
%\begin{align}
%    K_{sp}(X,Y) &= \sum_{m=1}^M \kappa^L(X_m, Y_m), \ where \\
%    \kappa^L(X_m,Y_m) &= 2^{-L}I^{l}(X_m,Y_m) +\sum_{l=1}^L 2^{-L+l-1}I^{l}(X_m,Y_m) \label{kappa}\\
%    I^{l}(X_m,Y_m) = \sum_{i=1}^{4^l}\min()
%\end{align}



%% Local Variables:
%% TeX-master: "main_eccv"
%% End: